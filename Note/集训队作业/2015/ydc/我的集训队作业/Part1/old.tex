\documentclass{ctexart}
\usepackage{amsmath}
\usepackage{amssymb}
\usepackage{fancyhdr}
\usepackage{ifthen}
\usepackage{syntonly}
\usepackage[colorlinks, CJKbookmarks=true, linkcolor=red]{hyperref}
\pagestyle{fancy}
\begin{document}
	\section{codeforces 253E Printer}
		\subsection{题目翻译}
			\subsubsection{【描述】}
				我们考虑一个有这样功能的网络打印机:他从时刻0开始工作,每一秒钟他打印一页纸。某些时刻他会收到一些打印任务。我们知道打印机会收到n个任务,我们将它们分别编号为连续的整数$1\sim n$,并且第$i$个任务用三个参数描述:$t_i$表示接到的时间,$s_i$表示任务要求你打印多少张,以及$p_i$表示任务的一个优先级。所有任务的优先级互不相同。
				当一个打印机收到一个任务时,任务会进入一个队列并留下直到完成了这个任务为止。在任务队列非空时,每个时刻,打印机会选择队列里优先级最高的一个任务,打印一页。你可以想象任务进入队列是瞬间的事情,所以他可以在收到某个任务的时刻去执行这个任务。

				你会得到除了某个任务以外所有任务的信息:你不知道某个任务的优先级是多少。然而,我们还额外的知道这个任务他完成时的时刻。我们给你这些信息,请求出这个未知的优先级并对每个任务输出它完成时的时刻。
			\subsubsection{【输入】}
				第一行输入一个整数$n(1 \le n \le 50000)$。接下来$n$行描述一个任务。第$i$行有三个整数分别是$t_i,s_i,p_i(0\le t_i \le 10^9,1 \le s_i,p_i \le 10^9)$。有且仅有一个任务(我们不妨称其为任务$x$),满足$p_x=-1$。所有的优先级互不相同,最后一行包含一个整数$T$——任务$x$完成的时刻,$1 \le T \le 10^{15}$。$t_i$并不一定互不相同。这个$x$可以是输入的任意一个任务。

			\subsubsection{【输出】}
				在第一行输出一个整数$p_x$——即任务$x$的优先级(要求$1 \le p_x$,并请记住所有任务的优先级必须互不相同)。接下来输出$n$个数字,第$i$个数字表示第$i$个任务结束时的时间。

				我们保证数据有解。如果有多解,输出任意一组解即可。
	\newpage
	\section{codeforces 348 E Pilgrims}
		\subsection{题目翻译}
			\subsubsection{【描述】}
				在很久以前有一片土地被称为Dudeland。Dudeland包含$n$个城镇,它们用$n-1$条双向道路连接起来。这些城镇通过道路可以两两互达。这里有$m$个修道院坐落在$m$个不同的城镇。每个修道院有一个教徒。

				在一年之始,每个教徒会选择离他最远的一个修道院。如果有多个,他会把所有的都列入清单。在``Big Lebowski day''里,每个教徒会随机选择一个清单里的城镇开始走去。

				Walter讨厌教徒。他想尽可能的通过阻止他们的行程来让尽可能多的人不开心。他计划摧毁一个没有修道院的城镇。一个教徒如果在他的清单里没有任何一个城镇能去,他就会不开心。

				你需要求出Walter最多能让几个教徒不开心。除此之外,你还要计算他有多少种方法。
			\subsubsection{【输入】}
				第一行包含两个整数$n,m$,满足$3 \le n \le 10^5,2 \le m <n$。

				接下来一行,有$m$个互不相同的整数,他们代表了有修道院的城镇的编号。

				接下来$n-1$行,每行三个整数$a_i,b_i,c_i$,表示$a_i,b_i$之间有一条边权为$c_i$的边。($1 \le a_i,b_i \le n,a_i \not = b_i,c_i \le 1000$)

			\subsubsection{【输出】}
				输出两个数:最多能让几个教徒不开心,以及有多少种方式达到这种效果。
	\newpage
	\section{GCJ 2014 Final F}
		\subsection{题目翻译}
			\subsubsection{【描述】}
				在游戏英雄联盟里,你可以玩一种游戏叫``ARAM'',这是``All Random, All Mid''的缩写。这个题目和它有点相似,但并不需要你了解英雄联盟。

				每次你开始玩``ARAM'',你会被随机分配为一种``champion'',总共$n$种。有一些``champion''你能更加轻松地取胜。所以如果你不幸分到一个概率低的,你会想得到一个不同的``champion''。幸运的是,游戏包含"Reroll"功能。
F
				``Reroll''会重新将你随机分配为一种``champion'',但你不能任意时刻都``Reroll''。具体地说它需要耗费钱。在你玩``ARAM'' 游戏前,你一开始就有$R$的``RD''(``Reroll dollars'')。你能``Reroll''当且仅当你有至少1RD。进行一次``Reroll''操作会花费1RD。 每次游戏后,你都会得到$\frac{1}{G}$的RD($G$是个整数),但你不能得到超过$R$的RD。如果你有$R$的RD,再玩一盘,你仍然是$R$的RD。

				如果你有至少1RD,并且你选择``Reroll'',你会花费1RD并重新随机分配成$n$个``champion''中的一个。你有可能会和之前分配给同一个。如果你不喜欢这次``Reroll''所得到的,并且你还有至少1RD,你可以再进行一次``Reroll''。只要你有至少1RD,你就能继续``Reroll''。

				举个例子,如果$R=2,G=2$,并且你使用了一次reroll在你第一次游戏中,第一次游戏结束后你会有1.5RD。如果你又玩了一个游戏,没有用reroll,你会有2.0RD。如果你再玩一个游戏不用reroll,你仍然是2.0RD(因为任意时刻不允许超过R)。如果你使用了两次reroll在你下一次游戏中,你就会变成0.5RD。

				你有一张表,记录你用第$i$个``champion''的胜率。你会玩$10^{100}$盘游戏并选择一种最优秀的策略。求期望的取胜次数比例。
			\subsubsection{【输入】}
				第一行一个整数$T$表示数据组数。接下来$T,1 \le T \le 100$组数据,每组数据第一行输入三个整数$n,R,G,1 \le n \le 1000,1 \le R,G \le 20$。接下来一行$n$个实数$p_i,0.0 \le p_i \le 1.0$,表示获胜概率。$p_i$会是一个四位小数。

				由于原题是提交答案题,所以我把数据规模做了些调整,以上参数并不会同时取到最大值,具体数据情况将在清橙题面上特殊说明。
			\subsubsection{【输出】}
				对于每组数据,输出一个实数表示期望的取胜次数所占比例。如果你的答案与标准答案绝对误差或相对误差不超过$10^{-10}$,你就会被判定为正确。
\end{document}

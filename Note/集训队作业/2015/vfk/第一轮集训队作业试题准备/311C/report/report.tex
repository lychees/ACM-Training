\documentclass[11pt, a4paper]{article}
\usepackage[slantfont, boldfont]{xeCJK}
\usepackage{ulem}
\usepackage{amsmath}
\usepackage{booktabs}
\usepackage{colortbl}
\usepackage{fancyvrb}
\usepackage{multirow}
\usepackage[top = 1.0in, bottom = 1.0in, left = 1.0in, right = 1.0in]{geometry}

\setCJKmainfont{SimSun}
\setCJKmonofont{SimSun}

\setlength{\parskip}{0.5\baselineskip}
\setlength{\parindent}{2em}

\title{Codeforces. Fetch the Treasure}
\author{VFleaKing}

\begin{document}
\maketitle

\section*{题目描述}
Rainbow建造了$h$间密室排成一行,从左至右编号为$1$到$h$。这些密室中有$n$间密室里有宝藏。我们称这样的密室为``宝藏密室''。第$i$间(从$1$开始编号)宝藏密室编号为$a_i$,里面的宝藏价值$c_i$美元。

Freda从第一间密室出发探险。每次她可以向前走$k$间密室或者回到第一间密室。也就是说Freda只能到达第$1, k+1, 2k+1, 3k+1 \dots$间密室。

不过,Rainbow给了Freda $m$个操作。每个操作是下列三种类型之一:
\begin{enumerate}
  \item 增加技能$x$。即允许Freda每次前进$x$间密室。比如,初始时她只有一个技能$k$。如果某时刻她有技能$a_1, a_2, \dots, a_r$,那么她就可以到达所有密室编号能写成$1+\sum_{i=1}^{r}{v_i a_i}$的密室(其中对于每个$i$,$v_i$是某个非负整数)。
  \item 让第$x$间宝藏密室的宝藏价值减少$y$美元。换句话说执行$c_x = c_x - y$。
  \item 查询Freda能到达的密室中价值最大的宝藏并拿走。如果Freda不能到达任何一个宝藏密室那么就认为价值最大的宝藏的价值为$0$并不做任何事。否则拿走这个价值最大的宝藏。如果有多个宝藏密室里的宝藏的价值达到最大,那么取走宝藏密室编号最小的宝藏密室(不必是密室编号最小的)。之后宝藏密室的数量减少一个。
\end{enumerate}

作为一个程序员,Freda请你写一个程序来回答每次询问。

\section*{输入格式}
输入的第一行包含四个整数$h, n, m, k$,$(h, n, m, k \geq 1)$。

接下来$n$行每行包含两个整数$a_i, c_i$,$(1 \leq a_i \leq h, 1 \leq c_i \leq 10^9)$,表示第$i$间宝藏密室是第$a_i$间密室,里面的宝藏价值$c_i$美元。所有的$a_i$互不相同
。

接下来$m$行每行是下面三种形式之一:
\begin{itemize}
  \item \texttt{1 x} --- 进行1号操作。$1 \leq x \leq h$。
  \item \texttt{2 x y} --- 进行2号操作。$1 \leq x \leq n, 0 \leq y < c_x$。
  \item \texttt{3} --- 进行3号操作。
\end{itemize}

最多只会有$20$个1号操作。保证任何时刻宝藏密室里的宝藏价值都是正数。保证所有的操作都是合法的(不会减少已被拿走宝藏的宝藏房间里的宝藏价值)。

\section*{输出格式}
对于每个操作3,输出一个整数表示Freda能到达的密室中价值最大的宝藏的价值(单位为美元)。

\section*{样例}
\begin{Verbatim}[frame=single, label=input]
10 3 5 2
5 50
7 60
8 100
2 2 5
3
1 3
3
3

\end{Verbatim}

\begin{Verbatim}[frame=single, label=output]
55
100
50

\end{Verbatim}

\section*{数据范围}
\begin{center}
\begin{tabular}{|c|p{70pt}|p{70pt}|p{70pt}|p{70pt}|}
\hline
\# & $h$                             & $n$                          & $m$                            & $k$                          \\
\hline
1  & \multirow{4}{*}{$\leq 100$}     & \multirow{4}{*}{$\leq 50$}   & \multirow{4}{*}{$\leq 100$}    & \multirow{4}{*}{$\leq 100$}  \\
\cline{1-1}
2  &                                 &                              &                                &                              \\
\cline{1-1}
3  &                                 &                              &                                &                              \\
\cline{1-1}
4  &                                 &                              &                                &                              \\
\hline
5  & \multirow{6}{*}{$\leq 10^{18}$} & \multirow{6}{*}{$\leq 10^5$} & \multirow{6}{*}{$\leq 10^5$}   & \multirow{6}{*}{$= 1$}       \\
\cline{1-1}
6  &                                 &                              &                                &                              \\
\cline{1-1}
7  &                                 &                              &                                &                              \\
\cline{1-1}
8  &                                 &                              &                                &                              \\
\cline{1-1}
9  &                                 &                              &                                &                              \\
\cline{1-1}
10 &                                 &                              &                                &                              \\
\hline
11 & \multirow{7}{*}{$\leq 10^{18}$} & \multirow{7}{*}{$\leq 100$}  & \multirow{7}{*}{$\leq 100$}    & \multirow{7}{*}{$\leq 10$}   \\
\cline{1-1}
12 &                                 &                              &                                &                              \\
\cline{1-1}
13 &                                 &                              &                                &                              \\
\cline{1-1}
14 &                                 &                              &                                &                              \\
\cline{1-1}
15 &                                 &                              &                                &                              \\
\cline{1-1}
16 &                                 &                              &                                &                              \\
\cline{1-1}
17 &                                 &                              &                                &                              \\
\hline
18 & \multirow{3}{*}{$\leq 10^{18}$} & \multirow{3}{*}{$\leq 10^5$} & \multirow{3}{*}{$\leq 10^5$}   & \multirow{3}{*}{$\leq 10^4$} \\
\cline{1-1}
19 &                                 &                              &                                &                              \\
\cline{1-1}
20 &                                 &                              &                                &                              \\
\hline
\end{tabular}
\end{center}

时间限制:$1\text{s}$

空间限制:$256\text{MB}$

\section*{题解}
用字母$n_x$表示技能的个数。

\subsection*{算法一}
我们可以直接根据题意进行模拟。对于操作一增加一个技能之后暴力bfs求出可到达的密室,这一步是$O(n_x h)$的,对于操作二$O(1)$进行操作,对于操作三每次暴力$O(n)$找到最优解。

总时间复杂度$O(nm + n_x^2 h)$。可以通过前4个测试点获得20分。

\subsection*{算法二}
对于5到10号点,$k = 1$,这意味着所有密室都是可达的。

那么我们可以无视所有的操作一,关注操作二和操作三。我们发现可以很轻松的用堆解决问题。操作二即减小一个键,操作三即取出堆顶元素,这两步都是$O(\log n)$的。

时间复杂度$O(m \log n)$,可以通过5到10号点获得30分。

\subsection*{算法三}
对于11号到17号点,$n, m, k$的范围都很小,特别是$k$最多为$10$。

初始状态下模$k$余$1$的密室都是可达的。注意到一个显然的性质:若$p$可达,那么$p + k$必然可达。这意味着可不可达与模$k$的余数有密切关系,我们可以对于每一种余数$r$,记录下可达的密室中编号最小的,那么判定也就容易了。

模$k$余$r$的可达密室中编号最小的求法显然我们可以用最短路解决。我们可以构一个结点数为$k$的图,结点编号从$0$到$k - 1$。对于每个技能$x$,我们对于所有结点$i$,连一条长度为$x$的边到结点$(i + x) \bmod k$。然后跑一遍最短路就行了。求最短路可以直接用$Floyd$。

对于操作二和操作三可以使用暴力。

时间复杂度为$O(nm + n_x (k n_x + k^3 + n))$。可以通过11号到17号点获得35分。

\subsection*{算法四}
把算法二和算法三结合起来。

由于对于所有数据$k \leq 10^4$,所以我们不可以再Floyd了,用Dijkstra就好了。操作二三还是使用堆。

时间复杂度为$O(m \log n + n_x(k n_x \log k + n))$。可以通过所有测试点获得100分。

\end{document}
